\documentclass[10pt, a4paper]{article}
\usepackage[paper=a4paper, left=1.5cm, right=1.5cm, bottom=1.5cm, top=3.5cm]{geometry}
\usepackage[latin1]{inputenc}
\usepackage[T1]{fontenc}
\usepackage[spanish]{babel}
\usepackage{indentfirst}
\usepackage{fancyhdr}
\usepackage{latexsym}
\usepackage{lastpage}
\usepackage{aed2-symb,aed2-itef,aed2-tad}
\usepackage[colorlinks=true, linkcolor=blue]{hyperref}
\usepackage{calc}
\usepackage{graphicx}
\usepackage[tight]{subfigure}
\usepackage{ifthen}

\newcommand{\f}[1]{\text{#1}}
\renewcommand{\paratodo}[2]{$\forall~#2$: #1}

\sloppy

\hypersetup{%
 % Para que el PDF se abra a p�gina completa.
 pdfstartview= {FitH \hypercalcbp{\paperheight-\topmargin-1in-\headheight}},
 pdfauthor={C�tedra de Algoritmos y Estructuras de Datos II - DC - UBA},
 pdfkeywords={Tp 1},
 pdfsubject={Especificacion de TADs}
}

\parskip=5pt % 10pt es el tama�o de fuente

% Pongo en 0 la distancia extra entre �temes.
\let\olditemize\itemize
\def\itemize{\olditemize\itemsep=0pt}

% Acomodo fancyhdr.
\pagestyle{fancy}
\thispagestyle{fancy}
\addtolength{\headheight}{1pt}
\lhead{Algoritmos y Estructuras de Datos II}
\rhead{$1^{\mathrm{er}}$ cuatrimestre de 2017}
\cfoot{\thepage /\pageref{LastPage}}
\renewcommand{\footrulewidth}{0.4pt}

\author{Algoritmos y Estructuras de Datos II, DC, UBA.}
\date{}
\title{Tp 1}

\begin{document}



%TADS
\section{Renombres de TADs}

\tadNombre{TAD Casillero} es \tadNombre{String}

\tadNombre{TAD Continente} es \tadNombre{conj(Casillero)}

\tadNombre{TAD Jugador} es \tadNombre{String}

\section{TAD \tadNombre{Tablero}}

\begin{tad}{\tadNombre{Tablero}}
\tadGeneros{tab}
\tadExporta{tab, observadores, casilleros, tableroValido?}

\tadIgualdadObservacionalSimple{(\paratodo{tab}{t1,t2}) (continentes(t1)= continentes(t2) \land movimientos(t1)= movimientos(t2))}
\tadObservadores

\tadOperacion{continentes}{tab}{conj(Continente)}{}
\tadOperacion{movimientos}{tab}{conj(tupla(Casillero, Movimiento, Casillero))}{}

\tadGeneradores
\tadOperacion{tableroVac�o}{}{tab}{}
\tadOperacion{agregarContinente}{Continente /c, tab /t}{tab}{c \cap casilleros(t) = \emptyset}
\tadOperacion{agregarMovimiento}{Casillero /p, Movimiento /m, Casillero /q, tab /t}{tab}{p \in casilleros(t) \land q \in casilleros(t) \land p\neq q \land \neg m \in movimientosDesde(p,t) }
\tadOtrasOperaciones
\begin{framed}
    \begin{itemize}
        \item continentesConexos: verifica s� se cumple la condici�n "`Agrupamiento de Continentes"'.
				\item movimientosDede: se usa en el generador agregarMovimiento para garantizar la Unicidad de Movimientos.
    \end{itemize}
\end{framed}

\tadOperacion{casilleros}{tab}{conj(casilleros)}{}
\tadOperacion{conexo?}{tab}{bool}{}
\tadOperacion{continentesConexos?}{tab}{bool}{}
\tadOperacion{movimientosDesde}{Casillero,tab}{conj(Movimiento)}{}
\tadOperacion{simetria?}{tab}{bool}{}
\tadOperacion{tableroValido?}{tab}{bool}{}

\begin{framed}
    \begin{itemize}
        \item Estas operaciones solo se usan como auxiliares para la axiomatiaci�n de las pimeras seis.
				\item est�nConectados: dado dos conjuntos de casilleros, es true s� todo elemento del primero esta conectado con todos los del segundo.
				\item todosALosQuePuedoLlegarDesde: dado un casillero "`p"' es el conjunto de casilleros "`q"'  tal que se puede llegar de p a q. 
				\item conectadosA: son todos con los que el casillero se puede conectar a treves de un solo movimiento.
				\item conectadosDeConectadosA: dado un conj de casilleros, es la union  de "`aplicar"' conectadosA a cada elemento del conjunto.
    \end{itemize}
\end{framed}

\tadOperacion{unirContinentes}{conj(Continente)}{conj(Casillero)}{}
\tadOperacion{est�nConectados}{conj(Casillero) \ ps, conj(Casillero) \ qs,tab \ t}{bool}{ps \subseteq casilleros(t) \land qs \subseteq casilleros(t)}
\tadOperacion{todosALosQuePuedoLlegarDesde}{Casillero \ p,tab \ t}{conj(Casillero)}{p \in casilleros(t)}
\tadOperacion{todosALosQuePuedoLlegarDesdeAux}{conj(Casillero) \ ps,tab \ t}{conj(Casillero)}{ps \subseteq casilleros(t)}
\tadOperacion{conectadosA}{Casillero, tab}{conj(Casillero)}{}
\tadOperacion{conectadosDeConectadosA}{conj(Casillero), tab}{conj(Casillero)}{}
\tadOperacion{continentesConexos?Aux}{conj(Continente),tab}{bool}{}
\tadOperacion{recortarTab}{Continente, tab}{tab}{}
\tadOperacion{simetria?Aux}{conj(casillero), tab}{bool}{}
\tadOperacion{simetria?Aux2}{casiillero,conj(casillero),tab}{bool}{}


\tadAxiomas[\paratodo{Casillero}{p, q, r}, \paratodo{conj(Casillero)}{ps, qs}, \paratodo{Continente}{c}, \paratodo{conj(Continente)}{cs}, \paratodo{Movimiento}{m}, 
\paratodo{tab}{t}]


\tadAxioma{continentes(tableroVac�o)}{\emptyset}
\tadAxioma{continentes(agregarContinentes(c,t))}{Ag(c,continentes(t))}
\tadAxioma{continentes(agregarCasillero(p,c,t))}{continentes(t)}
\tadAxioma{continentes(agregarMovimiento(p,m,q,t))}{continentes(t)}
\tadAxioma{movimientos(tableroVac�o)}{\emptyset}
\tadAxioma{movimientos(argregarContinentes(c,t))}{movimientos(t)}
\tadAxioma{movimientos(agregarMovimientos(p,m,q,t))}{Ag(<p,m,q>,movimientos(t))}
\tadAxioma{casilleros(t)}{unirContinentes(continentes(t))}
\tadAxioma{unirContinentes(cs)}{\IF $vacio?$ THEN $\emptyset$ ELSE $dameUno(cs) \cup unirContinentes(sinUno(cs))$ FI}
\tadAxioma{conexo?(t)}{est�nConectados(casilleros(t),casilleros(t),t)}
\tadAxioma{est�nConectados(ps,qs,t)}{ \IF $vacia?(ps)$ THEN $true$ ELSE $todosALosQuePuedoLlegarDesde(dameUno(ps),t) = qs \land estanConectados(sinUno(ps),qs,t)$FI}
\tadAxioma{todosALosQuePuedoLlegarDesde(p,t)}{todosALosQuePuedoLlegarDesdeAux(Ag(p, \emptyset),t)}
\tadAxioma{todosALosQuePuedoLlegarDesdeAux(ps,t)}{\IF $conectadosDeConectadosA(ps,t) \subseteq ps$ THEN $ps$ ELSE $todosALosQuePuedoLlegarDesdeAux(conectadosDeConectadosA(ps,c),t)$ FI}
\tadAxioma{conectadosDeConectadosA(ps,t)}{\IF $vacia?(ps)$ THEN $\emptyset$ ELSE $conectadosA(dameUno(ps),c) \cup dameUno(ps) \cup conectadosDeConectadosA(sinUno(us),c)$ FI}
\tadAxioma{conectadosA(p,tableroVac�o)}{\emptyset}
\tadAxioma{conectadosA(p,agregarContinente(c,t)}{conectadosA(p,t)}
\tadAxioma{conectadosA(p,agregarMovimiento(q,m,r,t)}{\IF $p=q$ THEN $Ag(r,\emptyset) \cup conectadosA(p,t)$ ELSE $conectadosA(p,t)$ FI}
\tadAxioma{continentesConexos?(t)}{continentesConexos?Aux(continentes(t),t)}
\tadAxioma{continentesConexos?Aux(cs,t)}{\IF $vacio?(cs)$ THEN $true$ ELSE $conexo?(recortarTad(dameUno(cs),t)) \land continentesConexos?(sinUno(cs),t)$ FI}
\tadAxioma{recortarTab(c,tableroVac�o)}{tableroVac�o}
\tadAxioma{recortarTab(c,agregarContinente(d,t))}{\IF $c=d$ THEN $agregarContinente(c,recortarTab(c,t))$ ELSE $recortarTab(c,t)$ FI}
\tadAxioma{recortarTad(c,agregarMovimiento(q,m,r,t))}{\IF $p \in c \land q \in c$ THEN $agregarMovimiento(q,m,r,recortarTab(c,t))$ ELSE $recortarTab(c,t)$ FI}
\tadAxioma{movimientosDesde(p,tableroVac�o)}{\emptyset}
\tadAxioma{movimientosDesde(p,agregarContinente(c,t))}{movimientosDesde(p,t)}
\tadAxioma{movimientosDesde(p,agregarMovimiento(q,m,r,t))}{\IF $p = q$ THEN $Ag(m,movimientosDesde(p,t)$ ELSE $movimientosDesde(p,t)$ FI}
\tadAxioma{simetria?(t)}{simetria?Aux(casilleros(t),t)}
\tadAxioma{simetria?Aux(ps,t)}{\IF $vacio?(ps)$ THEN $true$ ELSE $ simetriaAux2(dameUno(ps),conectadosA(dameUno(ps),t),t) \land simetria?Aux(sinUno(ps),t)$ FI}
\tadAxioma{simetria?Aux2(p,qs,t)}{\IF $vacio?(qs)$ THEN $true$ ELSE $ p \in conectadosA(dameUno(qs)) \land simetria?Aux2(p,sinUno(qs),t) $ FI}
\tadAxioma{tableroValido?(t)}{conexo?(t) \land continentesConexos?(t) \land simetria?(t)}
\end{tad}

\end{document}